\documentclass{jsarticle}
\usepackage{amsmath}
\usepackage{amsthm}
\usepackage{amsfonts}
\usepackage{mathtools}
\usepackage{mathrsfs}
\usepackage{ascmac}
\usepackage{bm}
\usepackage{listings}
\theoremstyle{plain}
\newtheorem{dfn}{Definition}[subsection]
\title{ソフトウェア演習1 \\
第1回レポート課題}
\author{J2200071 齊藤 隆斗}
\date{\today} \begin{document}
\maketitle


\section{課題2.3}
正規表現の字句解析ルーチンを作成し、正規表現入力を字句解析して得られた、トークンと意味値(あるときのみ)
の列を表示するプログラムを作成せよ.





\section{実行結果}

実行例1: 正規表現の文字に対する字句解析が行えているかを確認
\begin{lstlisting}
	$ ./kadai1 '(a|b).c*\ed\0'
	LPAR
	LETTER(a)
	VERT
	LETTER(b)
	RPAR
	CONC
	LETTER(c)
	AST
	EPSILON
	LETTER(d)
	EMPTY
	EOREG
\end{lstlisting}

実行例2: エスケープシーケンスの挙動が期待通りであることを確認
\begin{lstlisting}
	$ ./kadai1 '\(ab\c\)\*\\\.\d'
	LETTER(()
	LETTER(a)
	LETTER(b)
	LETTER(c)
	LETTER())
	LETTER(*)
	LETTER(\)
	LETTER(.)
	LETTER(d)
	EOREG
\end{lstlisting}

実行例3: 最後の文字がエスケープシーケンスだった場合、そのエスケープシーケンスを無視して出力することを確認
\begin{lstlisting}
	$ ./kadai1 '(a|b)*c\'
	LPAR
	LETTER(a)
	VERT
	LETTER(b)
	RPAR
	AST
	LETTER(c)
	EOREG
\end{lstlisting}



\section{プログラムの流れ}
字句解析のプログラムの流れついて述べる. まず、文字列を1文字だけ読み込む.ここで、ヌル文字であった場合、EOREGを出力してプログラムを停止する.
そうでなく、読み込んだ文字列が対応する正規表現の入力文字であると判定された場合、その正規表現のトークンを出力し、存在するときのみ意味値も出力する.
しかし、対応する正規表現の入力文字が \textbackslash 0 や、\textbackslash e のようにエスケープシーケンスを含むような2文字であるものがある.
したがって、エスケープシーケンスが読み込まれたとき、通常と異なる制御を行う.具体的には、エスケープシーケンスが読み込まれたら、更にもう1文字読む.それが0であればEMPTY、eであればEPLSILON、
ヌル文字であればEOREG、それ以外の文字が読み込まれたら意味値がその読み込んだ文字であるようなトークンであるLETTER((読み込んだ文字))を出力する.
このエスケープシーケンスの処理においても、ヌル文字であった場合は処理が停止することに注意する.
同様の処理を入力文字列の終端に達するまで行う.




\section{考察}
考えたことや工夫したことは3つある. \\
\quad 1つ目は、入力文字に対する分岐をswitch文で工夫することによって実現したことである.
正規表現の対応する入力文字にマッチしているかどうかを判定する部分で、エスケープシーケンスを考えなければ、
正規表現に対応する入力文字は常に1文字であるため、サンプルプログラムのswitch文で対応することができる.
しかし、$\emptyset$や$\epsilon$に対応する入力文字列はそれぞれ、\textbackslash 0,\textbackslash eであり2文字となっている.
switchにおいてcaseのラベルは整数の定数でなくてはいけないため、1文字ずつしか読み込むことができない.
よって、サンプルプログラムのswitch文を同じように使うことによって分岐を実装することはできない.
そこで、1つ目のswitch文においてエスケープシーケンスにマッチした後、次の1文字を読み込み、さらにその文字に対して分岐処理を行うことで$\emptyset$や$\epsilon$の2文字の入力文字にも対応することできた.
つまり、エスケープシーケンスにマッチするcaseにおいて、さらにswitch文をネストすることで解決した. \\
\quad 2つ目は、文字列の最後にエスケープシーケンスが現れた際の分岐処理で簡潔なプログラムで処理できるようにしたことである.
文字列の最後にエスケープシーケンスが入力された場合にどのように処理するべきかについて考えた.
エスケープシーケンスを読み込んだ後にヌル文字を読み込んだ場合にEOREGを出力することで、
エスケープシーケンスが文字列の最後に来てしまった場合に、そのエスケープシーケンスを無視するという挙動にすることでプログラムを簡潔にすることができた. \\
\quad 3つ目は、文字列の読み込み方と今回のプログラムの相性について考えた.
Cのライブラリには文字列比較の関数があり、2文字の文字列の処理にこの関数を使用することを考えたが、文字列を1文字目から終端まで検討していく今回のプログラムでは
上で述べたようなswitch文をネストして分岐を制御するほうが相性が良いと考えた.
なぜならば、2文字の分岐はエスケープシーケンスが読み込まれたときのみであり、1文字目は常に\textbackslash であるため、残りの1文字の分岐が行えれば十分であり、
わざわざ文字列比較の関数を使用する必要はないからである.


\end{document}
