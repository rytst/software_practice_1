\documentclass{jsarticle}
\usepackage{amsmath}
\usepackage{amsthm}
\usepackage{amsfonts}
\usepackage{mathtools}
\usepackage{mathrsfs}
\usepackage{ascmac}
\usepackage{bm}
\usepackage{listings}
\theoremstyle{plain}
\newtheorem{dfn}{Definition}[subsection]
\title{ソフトウェア演習1 \\
第2回レポート課題}
\author{J2200071 齊藤 隆斗}
\date{\today} \begin{document}
\maketitle


\section{課題3.7\&3.8+応用課題3.9}
課題3.7 \\
算術式の構文解析ルーチンを参考にして正規表現の構文解析ルーチンを作成し、入力正規表現の構文木を表示するプログラムを作成せよ.
正規表現の*は、算術式の符号の場合と異なり、後置演算子であることに注意する必要がある. \\

課題3.8 \\
引数 -d0, -d1, -d2 を指定することによって、入力正規表現、トークン列、構文木の3つをそれぞれ表示できるように main 関数を変更せよ. \\

応用課題3.9 \\
通常の grep プログラムでは、連結の '.' は省略されるのが普通である.そのような場合、上の方法で正規表現の構文解析がうまく行く確かめよ.
うまくいかないとすれば、どのように方針を修正すれば良いだろうか? \\


\section{実行結果}

\subsection*{課題3.7が正常に動作することを確認}

実行例1: テキストの例について確認
\begin{lstlisting}
$ ./kadai2 'a.\|.b'
   LETTER(a)
CONC
      LETTER(|)
   CONC
      LETTER(b)
\end{lstlisting}

実行例2: テキストの例について確認
\begin{lstlisting}
$ ./kadai2 '(a|b*)|c'
      LETTER(a)
   VERT
         LETTER(b)
      AST
VERT
   LETTER(c)
\end{lstlisting}

実行例3: 追加の例についても確認
\begin{lstlisting}
$ ./kadai2 '((a|b)*|\e).\0'
            LETTER(a)
         VERT
            LETTER(b)
      AST
   VERT
      EPSILON
CONC
   EMPTY
\end{lstlisting}

実行例4: 右結合であることを確認
\begin{lstlisting}
$ ./kadai2 'abcd'
   LETTER(a)
CONC
      LETTER(b)
   CONC
         LETTER(c)
      CONC
         LETTER(d)
\end{lstlisting}


\subsection*{課題3.8が正常に動作することを確認}

実行例5: d0オプションの動作について確認
\begin{lstlisting}
$ ./kadai2 -d0 '(a|b*)|c'
(a|b*)|c
\end{lstlisting}

実行例6: d1オプションの動作について確認
\begin{lstlisting}
$ ./kadai2 -d1 '(a|b*)|c'
LPAR
LETTER(a)
VERT
LETTER(b)
AST
RPAR
VERT
LETTER(c)
EOREG
\end{lstlisting}

実行例7: d2オプションの動作について確認
\begin{lstlisting}
$ ./kadai2 -d2 '(a|b*)|c'
      LETTER(a)
   VERT
         LETTER(b)
      AST
VERT
   LETTER(c)
\end{lstlisting}

実行例8: 予期しないオプションに対するエラー処理について確認
\begin{lstlisting}
$ ./kadai2 -d3 '(a|b*)|c'
invalid option
\end{lstlisting}

\subsection{応用課題3.9が正常に動作することを確認}


実行例9: テキストの例について確認
\begin{lstlisting}
./kadai2 -d2 'ab*c'
   LETTER(a)
CONC
         LETTER(b)
      AST
   CONC
      LETTER(c)
\end{lstlisting}

実行例10: 追加の例について確認
\begin{lstlisting}
$ ./kadai2 -d2 'a(\0|\e)*|b\e'
      LETTER(a)
   CONC
            EMPTY
         VERT
            EPSILON
      AST
VERT
      LETTER(b)
   CONC
      EPSILON
\end{lstlisting}



\section{プログラムの流れ}



\section{考察}



\end{document}
