\documentclass{jsarticle}
\usepackage{amsmath}
\usepackage{amsthm}
\usepackage{amsfonts}
\usepackage{mathtools}
\usepackage{mathrsfs}
\usepackage{ascmac}
\usepackage{bm}
\usepackage{listings}
\theoremstyle{plain}
\newtheorem{dfn}{Definition}[subsection]
\title{ソフトウェア演習1 \\
第2回レポート課題}
\author{J2200071 齊藤 隆斗}
\date{\today} \begin{document}
\maketitle


\section{課題3.7\&3.8+応用課題3.9}
課題3.7 \\
算術式の構文解析ルーチンを参考にして正規表現の構文解析ルーチンを作成し、入力正規表現の構文木を表示するプログラムを作成せよ.
正規表現の*は、算術式の符号の場合と異なり、後置演算子であることに注意する必要がある. \\

課題3.8 \\
引数 -d0, -d1, -d2 を指定することによって、入力正規表現、トークン列、構文木の3つをそれぞれ表示できるように main 関数を変更せよ. \\

応用課題3.9 \\
通常の grep プログラムでは、連結の '.' は省略されるのが普通である.そのような場合、上の方法で正規表現の構文解析がうまく行く確かめよ.
うまくいかないとすれば、どのように方針を修正すれば良いだろうか? \\


\section{実行結果}

\subsection*{課題3.7が正常に動作することを確認}

実行例1: テキストの例について確認
\begin{lstlisting}
$ ./kadai2 'a.\|.b'
   LETTER(a)
CONC
      LETTER(|)
   CONC
      LETTER(b)
\end{lstlisting}

実行例2: テキストの例について確認
\begin{lstlisting}
$ ./kadai2 '(a|b*)|c'
      LETTER(a)
   VERT
         LETTER(b)
      AST
VERT
   LETTER(c)
\end{lstlisting}

実行例3: 追加の例についても確認
\begin{lstlisting}
$ ./kadai2 '((a|b)*|\e).\0'
            LETTER(a)
         VERT
            LETTER(b)
      AST
   VERT
      EPSILON
CONC
   EMPTY
\end{lstlisting}

実行例4: 右結合であることを確認
\begin{lstlisting}
$ ./kadai2 'abcd'
   LETTER(a)
CONC
      LETTER(b)
   CONC
         LETTER(c)
      CONC
         LETTER(d)
\end{lstlisting}


\subsection*{課題3.8が正常に動作することを確認}

実行例5: d0オプションの動作について確認
\begin{lstlisting}
$ ./kadai2 -d0 '(a|b*)|c'
(a|b*)|c
\end{lstlisting}

実行例6: d1オプションの動作について確認
\begin{lstlisting}
$ ./kadai2 -d1 '(a|b*)|c'
LPAR
LETTER(a)
VERT
LETTER(b)
AST
RPAR
VERT
LETTER(c)
EOREG
\end{lstlisting}

実行例7: d2オプションの動作について確認
\begin{lstlisting}
$ ./kadai2 -d2 '(a|b*)|c'
      LETTER(a)
   VERT
         LETTER(b)
      AST
VERT
   LETTER(c)
\end{lstlisting}

実行例8: 予期しないオプションに対するエラー処理について確認
\begin{lstlisting}
$ ./kadai2 -d3 '(a|b*)|c'
invalid option
\end{lstlisting}

\subsection*{応用課題3.9が正常に動作することを確認}


実行例9: テキストの例について確認
\begin{lstlisting}
./kadai2 -d2 'ab*c'
   LETTER(a)
CONC
         LETTER(b)
      AST
   CONC
      LETTER(c)
\end{lstlisting}

実行例10: 追加の例について確認
\begin{lstlisting}
$ ./kadai2 -d2 'a(\0|\e)*|b\e'
      LETTER(a)
   CONC
            EMPTY
         VERT
            EPSILON
      AST
VERT
      LETTER(b)
   CONC
      EPSILON
\end{lstlisting}



\section{プログラムの流れ}
大まかなプログラムの流れは次のようになる. \\
はじめに、与えられた引数の検証を行う.引数が存在しなかったり、引数がオプションと正規表現の2つよりも多くの引数が入力されてしまっている場合、
そして、不当なオプションが与えられている場合はエラー処理を行う. オプションについて、d0の場合は入力した正規表現をそのまま出力しプログラムを停止する、
d1の場合は入力した正規表現のトークン列を出力し停止する.d2の場合は与えられた正規表現に対する構文木を出力し停止する. オプションが入力されなかったときはデフォルトの処理として、
与えられた正規表現に対する構文木を出力する. \\
オプションがd0またはd1であったときについてのプログラムの流れについて、d0は正規表現を出力するだけ、d1は前回の課題で扱ったlexer()関数を実行するだけであるから省略する.
d2または、オプションが与えられなかったとき、すなわち構文木を出力する場合についてのプログラムの流れについて大まかに見ていく.
このサブルーチンは、一文字ずつ入力を読み込んでいき構文規則にしたがって構文木を生成していく.生成の仕方は考察のセクションで触れる.構文木を正常に生成し終われば、
その構文木を出力して処理を停止する.



\section{考察}
この課題で考えたこと、工夫した点は2つある. \\
1つ目は、サブルーチンparse()の再帰的な実装である.
再帰を用いず、手続き的に処理を行うとすると、VERTやCONCなどの二項演算子の場合や、ASTなどの被演算子を前半に取るような一項演算子では、
構文木を作成する際に文字を読み込んだ後にもう一度もどって処理を行う必要がある.
例えばASTの場合では、ASTをルートに取る必要があるので、ASTを読み込みルートに設定したうえでASTの被演算子をそのルートの子に設定する.
今回は再帰を用いており読み込んだ文字を逆行するようなことはせず、サブルーチンを再帰的に呼び出すことによって構文木を生成していく.
例としてa$|$bについて考える.まず、aを読み込み、eval\_primary()によって処理が行われ、構文木のルートにaが格納される.
このeval\_primary()によって、$|$が読み込まれていて、再帰的に呼び出しているので呼び出し元の関数に戻っていき、条件文によって引っかかり、実行されるのは、
eval\_term()である.この関数は今まで作成した木をCONCをルートとした構文木において、CONCの左部分木として格納する.同様にこの関数によって、
次の文字は呼び出されているのでbを再帰的に処理をするとLETTER(b)をルートとする子を持たない木が作成され、
それを先程作成したCONCをルートとする右部分木に格納する.そして、次のトークンがEOREGとなったので処理をparse()を終了する.
このようにスタックを利用し、上位のサブルーチンを保留しておくことで再帰的に構文木を作成することができる. \\
2つ目は、応用課題3.9における'.'を省略しても連結として処理できるようにするプログラムの拡張である.
実装のアイデアは、CONCの直後に現れるトークンは、EPMTY,EPSILON,LETTERまたはLPARのいずれかであるから、eval\_term()関数において、
curr\_tokenがCONCでない場合について、これらのEMPTY,EPSILON,LETTERまたはLPARのいずれかの場合について、CONCと同様の処理を行えば良いというアイデアである.
CONCであった場合の処理については、get\_token()によってcurr\_tokenをCONCからつぎの文字に移している.しかし、今回はCONCの直後に現れるトークンがすでに読み込まれていて、
curr\_tokenに格納されている状態であるから、get\_token()関数を呼び出す必要がないことに注意する.



\end{document}
